\documentclass[10pt, conference]{IEEEtran} %compsocconf

\input{preamble1}

\begin{document}
%
% paper title
% can use linebreaks \\ within to get better formatting as desired
\title{Common Subexpression Elimination with Subtree Isomporphisms}


\author{\IEEEauthorblockN{Robert King(student)\IEEEauthorrefmark{1}, Milinda Fernando(mentor)\IEEEauthorrefmark{2},
Hari Sundar (mentor)\IEEEauthorrefmark{3}, }
\IEEEauthorblockA{
%School of Computing,~~~~~~~~~~~~~~~~~~~~~~~~~~~~~~~~~~~~~~~~~~~~~~~~~~~~~~~~~ Department of Physics,\\
%University of Utah, ~~~~~~~~~~~~~~~~~~~~~~~~~~~~~~~~~~~~~~~~~~~~~~~~~~~~~~~~~ Brigham Young University,\\
%Salt Lake City, Utah, ~~~~~~~~~~~~~~~~~~~~~~~~~~~~~~~~~~~~~~~~~~~~~~~~~~~~~~ Provo, Utah\\
\IEEEauthorrefmark{1}giskard.king@gmail.com,
\IEEEauthorrefmark{2}milinda@cs.utah.edu,
\IEEEauthorrefmark{3}hari@cs.utah.edu,
}}

%\author{\IEEEauthorblockN{Milinda Fernando (student)}
%\IEEEauthorblockA{School of Computing,\\
% University of Utah\\
% Salt Lake City, Utah\\
% Email: milinda@cs.utah.edu}
% \and
% \IEEEauthorblockN{David Nielsen}
% \IEEEauthorblockA{ Department of Physics, \\
% Brigham Young University\\
% Provo, Utah\\
% Email: david.nielsen@byu.edu}
% \and
% \IEEEauthorblockN{Hari Sundar (advisor)}
% \IEEEauthorblockA{School of Computing,\\
% University of Utah\\
% Salt Lake City, Utah\\
% Email: hari@cs.utah.edu}
% }

% conference papers do not typically use \thanks and this command
% is locked out in conference mode. If really needed, such as for
% the acknowledgment of grants, issue a \IEEEoverridecommandlockouts
% after \documentclass

% for over three affiliations, or if they all won't fit within the width
% of the page, use this alternative format:
% 
%\author{\IEEEauthorblockN{Michael Shell\IEEEauthorrefmark{1},
%Homer Simpson\IEEEauthorrefmark{2},
%James Kirk\IEEEauthorrefmark{3}, 
%Montgomery Scott\IEEEauthorrefmark{3} and
%Eldon Tyrell\IEEEauthorrefmark{4}}
%\IEEEauthorblockA{\IEEEauthorrefmark{1}School of Electrical and Computer Engineering\\
%Georgia Institute of Technology,
%Atlanta, Georgia 30332--0250\\ Email: see http://www.michaelshell.org/contact.html}
%\IEEEauthorblockA{\IEEEauthorrefmark{2}Twentieth Century Fox, Springfield, USA\\
%Email: homer@thesimpsons.com}
%\IEEEauthorblockA{\IEEEauthorrefmark{3}Starfleet Academy, San Francisco, California 96678-2391\\
%Telephone: (800) 555--1212, Fax: (888) 555--1212}
%\IEEEauthorblockA{\IEEEauthorrefmark{4}Tyrell Inc., 123 Replicant Street, Los Angeles, California 90210--4321}}
% use for special paper notices
%\IEEEspecialpapernotice{(Invited Paper)}
% make the title area
\maketitle


%\begin{abstract}
%We present a portable and highly-scalable algorithm and framework that
%targets problems in the astrophysics and numerical relativity
%communities. 
%This framework combines 
%together the parallel \dendro~ octree with wavelet adaptive 
%multiresolution and a physics module to solve the Einstein equations of 
%general relativity in the \BSSN~formulation.
%%Wavelet adaptive multiresolution is used to provide local, 
%%fine-grain adaptivity, based on the fast wavelet transform of interpolating
%%wavelets.
%The goal of this work is to perform advanced, massively parallel 
%numerical simulations of Intermediate Mass Ratio Inspirals (IMRIs)
%of binary black holes with mass ratios on the order of 100:1. These studies
%will be used to 
%generate waveforms for use in LIGO data analysis and to calibrate
%semi-analytical approximate methods. This advanced framework is designed 
%to easily accommodate many existing algorithms in astrophysics
%for 
%%compressible 
%plasma dynamics and radiation hydrodynamics. 
%We have designed novel algorithms to enable efficient simulations for such 
%experiments and demonstrate 
%excellent weak scalability up to $131K$ cores on ORNL's Titan for binary mergers for mass ratios up to $100$. 
%
%
%\end{abstract}

%\begin{IEEEkeywords}
% Numerical Relativity; Einstein equations; Adaptive Mesh Refinement; Finite Differencing; Wavelet Adaptive Multiresolution
%\end{IEEEkeywords}

% For peer review papers, you can put extra information on the cover
% page as needed:
% \ifCLASSOPTIONpeerreview
% \begin{center} \bfseries EDICS Category: 3-BBND \end{center}
% \fi
%
% For peerreview papers, this IEEEtran command inserts a page break and
% creates the second title. It will be ignored for other modes.
\IEEEpeerreviewmaketitle

\section{Introduction}
The purpose of this work is to improve the run time of the simulation of black hole collisions and their corresponding gravitational waves. Black hole collisions can be modeled using the BSSN equations. The BSSN equations consist of several complex partial differential equations and to model these equations the value of each variable is computed once per a time step in the model. The overall model is constructed by calculating millions of timesteps to see how black holes interact with each other. Each timestep must solve the partial differential equations. This process is automated using the python package SymPy. SymPy takes mathematical expressions and generates python code to solve each expression. However due to the complexity of the BSSN differential equations, the auto generated code consists of thousands of temporary variables. Due to the number of temporary variables, modern compilers are unable to effectively optimize the code causing the code to become incredibly inefficient. This thesis illustrates a technique to use Subtree Isomorphisms and Common Subexpression Elimination to improve the run time.
The focus of this work is to use a bottom up approach to find an efficient way to solve for the values of the partial differential equations for each timestep. The strategy is to convert all temporary variable computations into expression trees. Once the expression trees are created, they are rebuilt to take advantage of caching for the targeted memory architecture. In addition to outlining an algorithm to improve calculation, this work proves a lower bound for the registers necessary to calculate each temporary variable using graph partitioning.


\noindent The key \ul{contributions} of this work are:
%\begin{enumerate}
%    \item

\vspace{0.1in}    
%\item
\noindent \textbf{Automatic symbolic code-generation}. Given the complexity of the Einstein equations, we have developed an automatic code generation framework for GR using \texttt{SymPy} that automatically generates architecture-optimized codes. This greatly improves code portability, use by domain scientists and the ability to add additional constraints and checks to validate the code.


\section{Methodology}
The Figure 3 on the poster presents an overview of our approach. 
\vspace{-0.15in}


%\subsection{Octree partitioning, construction \& 2:1 balancing }
%In this work we use octree based adaptive grids, where the adaptivity is determined by wavelet coefficients of the solution represented. We use our previous work on Space Filling Curve(SFC) based flexible octree partitioning algorithm \tsort\cite{Fernando:2017} to perform parallel octree partitioning. We refer \emph{octree construction} as completion of an incomplete octree which is performed by top down traversal over the octant based on the SFC ordering specified. We impose 2:1 balancing constraint by using a similar approach to \cite{SundarSampathBiros08} which enforces that for any given octant in the octree, all its neighboring octant differ by refinement level $\leq 1$ . The 2:1 balancing constraint makes the subsequent steps such as mesh generation, \unzip~ \& \zip~ operations simpler. 


The research for this thesis consists of two main projects. The first is the subtree isomorphism problem that will focus on the common subexpression elimination and the second is a lower bound analysis for the number of temporary variables needed to solve the partial differential equations. The goal is to create an algorithm that will be able to analyze the different partial differential equations and reorder the temporary variable calculations to maximize cache effectiveness and variable reuse. 

\subsection{Staging}
	Staging is focused on finding variable reuse within the partial differential equations. The first problem is to create an expression tree from the partial differential equations generated from the SymPy auto generated code. 
	Once the expression tree is created, subtree isomorphism analysis can begin. This will be a bottom up approach that considers the values used in each leaf node in addition to finding similar tree structure. Each node within the tree will keep track of all the leaf values that it depends on. Set similarity will be used as a precondition before calculating the more expensive tree isomorphism. By the end of the Staging Process the most common subexpressions will be identified. Each expression will be valued depending on the number of leaf node dependents, to mitigate the total number of temporary variables, and the number of times each expression appears, to maximize data reuse.
	
\subsection{Rebuilding}
	The rebuilding phase takes the results from the staging process and rebuilds the expression tree. In order of importance the rebuilding phase must preserve the correct final answers, maximize cache effectiveness, and minimize the total number of temporary variables used. The current strategy is to use a dynamic programming approach. Dynamic programming will also allow for the rebuilding process to scale effectively. Once complete, these results will be measured experimentally. Once a proof of concept has been completed the rebuilding process will take in parameters such as memory architecture, and L1 cache size to optimize the calculations for each machine. Once the rebuilding process is completed a parser will be created to transform the original SymPy autogenerated code to reflect the updated expression tree.







%\begin{algorithm}
%	\caption{\small Overview of our approach}\label{alg:overview}
%	\footnotesize
%	\begin{algorithmic}[1]
%		%\Require A list of points or regions $W$, the starting level $l_1$ and the ending level $l_2$, $K_{oct}$
%		%\Ensure $\tau_c$ \- ordered complete octree based on $W$
%		%\Function{TreeSort($W$, $l_1$, $l_2$)}{}
%		\State $M \leftarrow$ initialize mesh \Comment{meshing}
%		\State $u \leftarrow$ initialize variables $(M)$
%		\While{$t < T$}
%		\For{$r = 1:3$} \Comment{Runge-Kutta stages}
%		\State $B, \hat{u} \leftarrow \text{Unzip}(M, u)$ \Comment{\S\ref{sec:unzip_and_zip}}
%		\For{$ b \in B$} 
%		\State Compute derivatives \Comment Machine generated code 
%		\State Compute $\hat{u}_{rhs}(b)$ \Comment Machine generated code 
%		\EndFor % blocks
%		\State $u_{rhs} \leftarrow \text{Zip}(M, B, \hat{u}_{rhs})$ \Comment{\S\ref{sec:unzip_and_zip}}
%		\State RK update
%		\EndFor  % rk 
%		\State $t\leftarrow t+dt$
%		\If{need remesh $M$}  \Comment{remesh }
%		\State $M' \leftarrow$ remesh($M$) 
%		\State $u' \leftarrow$ Intergid\_Transfer$(M, M', u)$ \Comment{inter-grid transfer}
%		\EndIf
%		\EndWhile  % time
%	\end{algorithmic}
%\end{algorithm}


\vspace{-0.15in}
\section{Results}
As a proof of concept, an initial algorithm was developed to parse the SymPy autogenerated code into an encompassing expression tree and reduce the tree using a user specified cache size. The greedy strategy aims to create expression that with the number of dependencies equivalent to the specified cache size. This process is applied recursively to calculate all the target variables from the SymPy autogenerated code.
The preliminary results demonstrate that lower cache sizes cause parts of expressions tree to be duplicated. This similarity suggests there is an opportunity to effectively utilize memory locality to reduce run time. At larger cache sizes it is important to stage the expression tree to reduce the number of cache misses and improve performance.  

\begin{figure}[tbh]
	\centering
	\includegraphics[width=2.0\columnwidth]{Staging_Results.png}
	\caption{\small Preliminary Results	}
	\vspace{-0.1in}
\end{figure}


%============ Conclusion ===============
%\section{Conclusion}

%In the short time that LIGO and Virgo
%have been searching for
%gravitational waves, we have already learned exciting things about
%neutron stars~\cite{Most:2018hfd,Shibata:2017xdx}, 
%the production of heavy elements 
%(such as gold)~\cite{0004-637X-855-2-99}, 
%and the population of black holes in the 
%universe~\cite{TheLIGOScientific:2016htt}. 
%When gravitational wave observations are combined with observations of
%electromagnetic radiation---from radio waves to gamma rays---there is
%a multiplicative effect that magnifies the scientific impact. This is
%the promise of multi-messenger astronomy.
%
%The full scientific
%impact of multi-messenger astronomy is only realized when the
%observations are informed by sophisticated computer models of the underlying
%astrophysical phenomena. 
%\dendro\ provides the ability to run these models in a scalable way, 
%with local adaptivity criteria using WAMR.
%While AMR codes with block-adaptivity typically lose performance as the number
%of adaptive levels increases, \dendro\ achieves impressive scalability on a real 
%application even with many levels of refinement. The combination of scalability
%and adaptivity will allow us to study the gravitational radiation from
%IMRIs without simplifying approximations in direct numerical simulations.
%
%The \dendro\ code reported on here, with 
%a module for vacuum black hole spacetimes, is just our initial step in 
%creating a highly adaptive computational platform for studying
%relativistic astrophysics on the next-generation of supercomputers.
%This work will be followed with additional modules for solving the
%relativistic magnetohydrodynamics equations, 
%nuclear equations of state, and radiation hydrodynamics.
%For application developers, a key advantage of \dendro\ is the ability to use
%conventional numerical methods for these modules.
%
%As LIGO and Virgo are joined by other gravitational wave detectors and
%observatories around the world, we expect many more exciting 
%discoveries to come.




% conference papers do not normally have an appendix


% use section* for acknowledgement
%\section*{Acknowledgment}
%The authors would like to thank...
%more thanks here
% trigger a \newpage just before the given reference
% number - used to balance the columns on the last page
% adjust value as needed - may need to be readjusted if
% the document is modified later
%\IEEEtriggeratref{8}
% The "triggered" command can be changed if desired:
%\IEEEtriggercmd{\enlargethispage{-5in}}

% references section
%\IEEEtriggeratref{70}
% can use a bibliography generated by BibTeX as a .bbl file
% BibTeX documentation can be easily obtained at:
% http://www.ctan.org/tex-archive/biblio/bibtex/contrib/doc/
% The IEEEtran BibTeX style support page is at:
% http://www.michaelshell.org/tex/ieeetran/bibtex/
\bibliographystyle{IEEEtran}
\bibliography{bssn}


%\newpage
%\pagebreak
%\setcounter{page}{1}
%%\appendix
%\appendices
%
%\section{Artifact Description}
%\label{sec:AD}
%\subsection{Getting and Compiling \dendro~}

The \dendro\ simulation code is freely available at GitHub (\href{https://github.com/paralab/Dendro-GR}{https://github.com/paralab/Dendro-GR}) under the MIT License. 
The latest version of the code can be obtained by cloning the repository
\begin{lstlisting}[basicstyle=\small,language=bash]
$ git clone git@github.com:paralab/Dendro-GR.git
\end{lstlisting}

The following dependencies are required to compile \dendro~
\begin{itemize}
	\item C/C++ compilers with C++11 standards and OpenMP support
	\item MPI implementation (e.g. openmpi, mvapich2 )
	\item ZLib compression library (used to write \texttt{.vtu} files in binary format with compression enabled)
	\item BLAS and LAPACK are optional and not needed for current version of \dendro~
	\item CMake 2.8 or higher version
\end{itemize}

\textbf{Note}: We have tested the compilation and execution of \dendro\ with \texttt{intel}, \texttt{gcc} 4.8 or higher, \texttt{openmpi}, \texttt{mpich2} and \texttt{intelmpi} and \texttt{craympi} (in \Titan) using the linux operating systems. 

To compile the code, execute these commands
\begin{lstlisting}[language=bash]
$ cd <path to DENDRO directory >
$ mkdir build
$ cd build
$ ccmake ../     
\end{lstlisting}
The following options for \dendro\  can then be set in cmake:
\begin{itemize}
	\item \texttt{DENDRO\_COMPUTE\_CONSTRAINTS} : Enables the computation of Hamiltonian and momentum constraints
	\item \texttt{DENDRO\_CONSEC\_COMM\_SELECT}  : If \texttt{ON} sub-communicators are selected from consecrative global ranks, otherwise sub-communicators are selected complete binary tree of global ranks (note that in this case global communicator size need to a power of 2).
	\item \texttt{DENDRO\_ENABLE\_VTU\_CONSTRAINT\_OUT} : Enables constraint variable output \linebreak while time-stepping 
	\item \texttt{DENDRO\_ENABLE\_VTU\_OUTPUT} : Enables evolution variable output while time-stepping
	\item \texttt{DENDRO\_VTK\_BINARY} : If \texttt{ON} vtu files are written in binary format, else ASCII format (binary format recommended).
	\item \texttt{DENDRO\_VTK\_ZLIB\_COMPRES} : If \texttt{ON} binary format is compressed (only effective if \texttt{DENDRO\_VTK\_BINARY} is \texttt{ON})
	\item \texttt{HILBERT\_ORDERING} : Hilbert SFC used if \texttt{ON}, otherwise Morton curve is used. (Hilbert curve is recommended to reduce the communication cost.)
	\item \texttt{NUM\_NPES\_THRESHOLD} : When running in large scale set this to $\sqrt{p}$ where $p$ number of mpi tasks for better performance.
	\item \texttt{RK\_SOLVER\_OVERLAP\_COMM\_AND\_COM} : If \texttt{ON} non blocking communication is used and enable overlapping of communication and computation \textit{unzip} (recommended option), otherwise blocking synchronized \textit{unzip} is used.  	
\end{itemize}

After configuring \dendro, generate the Makefile (use \texttt{c} to configure and \texttt{g} to generate). Then execute \texttt{make all} to build all the targets.  On completion, \bsolver~will be the main executable as related to this paper. 

\subsection{Getting Started: Running \bsolver}

\bsolver\  can be run as follows. 
\begin{lstlisting}[language=bash]
$ mpirun -np <number of mpi tasks>\
 ./bssnSolver \
<parameter file name>.par
\end{lstlisting}
Example parameter files can be found in \texttt{BSSN\_GR/pars/}. The following 
is an example parameter file for equal mass ratio binary inspirals. 
\begin{lstlisting}[basicstyle=\tiny]
	{
	"DENDRO_VERSION": 5.0,
	"BSSN_RESTORE_SOLVER":0,
	"BSSN_IO_OUTPUT_FREQ": 10,
	"BSSN_REMESH_TEST_FREQ": 5,
	"BSSN_CHECKPT_FREQ": 50,
	"BSSN_VTU_FILE_PREFIX": "bssn_gr",
	"BSSN_CHKPT_FILE_PREFIX": "bssn_cp",
	"BSSN_PROFILE_FILE_PREFIX": "bssn_r1",
	"BSSN_DENDRO_GRAIN_SZ": 100,
	"BSSN_ASYNC_COMM_K": 4,
	"BSSN_DENDRO_AMR_FAC": 1e0,
	"BSSN_WAVELET_TOL": 1e-4,
	"BSSN_LOAD_IMB_TOL": 1e-1,
	"BSSN_RK_TIME_BEGIN": 0,
	"BSSN_RK_TIME_END": 1000,
	"BSSN_RK_TIME_STEP_SIZE": 0.01,
	"BSSN_DIM": 3,
	"BSSN_MAXDEPTH": 12,
	"ETA_CONST": 2.0,
	"ETA_R0": 30.0,
	"ETA_DAMPING": 1.0,
	"ETA_DAMPING_EXP": 1.0,
	"BSSN_LAMBDA": {
	"BSSN_LAMBDA_1": 1,
	"BSSN_LAMBDA_2": 1,
	"BSSN_LAMBDA_3": 1,
	"BSSN_LAMBDA_4": 1
	},
	"BSSN_LAMBDA_F": {
	"BSSN_LAMBDA_F0": 1.0,
	"BSSN_LAMBDA_F1": 0.0
	},
	"CHI_FLOOR": 1e-4,
	"BSSN_TRK0": 0.0,
	"KO_DISS_SIGMA": 1e-1,
	"BSSN_BH1": {
	"MASS":0.48528137423856954,
	"X": 4.00000000e+00,
	"Y":0.0,
	"Z": 1.41421356e-05,
	"V_X": -0.00132697,
	"V_Y": 0.1123844,
	"V_Z": 0,
	"SPIN": 0,
	"SPIN_THETA":0,
	"SPIN_PHI": 0
	},
	"BSSN_BH2": {
	"MASS":0.48528137423856954,
	"X":-4.00000000e+00,
	"Y":0.0,
	"Z":1.41421356e-05,
	"V_X": 0.00132697,
	"V_Y": -0.1123844,
	"V_Z": 0,
	"SPIN": 0,
	"SPIN_THETA":0,
	"SPIN_PHI": 0
	}
	}
\end{lstlisting}
Here we list the key options for \bsolver\ with a short description.
\begin{itemize}
\item \texttt{BSSN\_RESTORE\_SOLVER} : Set $1$ to restore $RK$ solver from latest checkpoint. 
\item \texttt{BSSN\_IO\_OUTPUT\_FREQ} : IO (i.e. \texttt{vtu} files) output frequency
\item \texttt{BSSN\_CHECKPT\_FREQ} : Checkpoint file output frequency
\item \texttt{BSSN\_REMESH\_TEST\_FREQ} : Remesh test frequency (i.e. frequency in time steps that is being tested for re-meshing) 
\item \texttt{BSSN\_DENDRO\_GRAIN\_SZ} : Number of octants per core 
\item \texttt{BSSN\_ASYNC\_COMM\_K} : Number of variables that are being processed during an asynchronous \textit{unzip} ($<24$)
\item \texttt{BSSN\_DENDRO\_AMR\_FAC} : Safety factor for coarsening i.e. coarsen if and only if $W_c \leq AMR\_FAC \times WAVELET\_TOL$ where $W_c$ is the computed wavelet coefficient.  
\item \texttt{BSSN\_WAVELET\_TOL} : Wavelet tolerance for WAMR. 
\item \texttt{BSSN\_MAXDEPTH} : Maximum level of refinement allowed ($\leq 30$)
\item \texttt{KO\_DISS\_SIGMA} : Kreiss-Oliger dissipation factor for \BSSN~formulation
\item \texttt{MASS} : Mass of the black hole
\item \texttt{X} : $x$ coordinate of the black hole
\item \texttt{Y} : $y$ coordinate of the black hole
\item \texttt{Z} : $z$ coordinate of the black hole
\item \texttt{V\_X} : momentum of the black hole in $x$ direction
\item \texttt{V\_Y} : momentum of the black hole in $y$ direction
\item \texttt{V\_Z} : momentum of the black hole in $z$ direction
\item \texttt{SPIN} : magnitude of the spin of the black hole
\item \texttt{SPIN\_THETA} : magnitude of the spin of the black hole along $\theta$
\item \texttt{SPIN\_PHI} : magnitude of the spin of the black hole along $\phi$
\end{itemize}

\subsubsection{Generating your own parameters}
The intial data parameters for a black hole binary depend
on the total mass ($M=m1+m2$), the mass ratio $q$ and the separation distance $d$. These parameters are calculated  using the python script \texttt{BSSN\_GR/scripts/id.py}. The command to generate parameters for $q=10$, total mass $M=5$ and separation $d=16$ is 
\begin{lstlisting}
$ python3 id.py -M 5 -r 10 16
\end{lstlisting}
\begin{lstlisting}[basicstyle=\tiny]
----------------------------------------------------------------
PUNCTURE PARAMETERS (par file foramt)
----------------------------------------------------------------
"BSSN_BH1": {
"MASS":4.489529,
"X":1.454545,
"Y":0.000000,
"Z": 0.000014,
"V_X": -0.020297,
"V_Y": 0.423380,
"V_Z": 0.000000,
"SPIN": 0.000000,
"SPIN_THETA":0.000000,
"SPIN_PHI": 0.000000
},
"BSSN_BH2": {
"MASS":0.398620,
"X":-14.545455,
"Y":0.000000,
"Z":0.000014,
"V_X": 0.020297,
"V_Y": -0.423380,
"V_Z": 0.000000,
"SPIN": 0.000000,
"SPIN_THETA":0.000000,
"SPIN_PHI": 0.000000
}
The tangential momentum is just an estimate, and the value for a
for a circular orbit is likely between (0.5472794147860968, 0.29947988193805547)
\end{lstlisting}

\subsection{Symbolic interface and code generation}
The \BSSN~formulation is a decomposition of the Einstein equations into 
24 coupled hyperbolic PDEs. Writing the computation code for the 
\BSSN~formulation can be a tedious task. Hence we have written a symbolic 
python interface to generate optimized C code to compute the \BSSN~equations. 
All the symbolic utilities necessary to write the \BSSN~formulation in symbolic 
python can be found in \texttt{GR/rhs\_scripts/bssn/dendro.py} and the symbolic \BSSN~code can 
be found in \texttt{GR/rhs\_scripts/bssn/bssn.py}. 
This could be modified for more advanced uses of the code such as including
new equations to describe additional physics or for introducing a different 
formulation of the Einstein equations.

\subsection{Profiling the code}
\dendro\ contains built-in profiler code which enables one to profile the 
code extensively. On configuration, a user can enable/disable 
the internal profiling flags using \texttt{ENABLE\_DENDRO\_PROFILE\_COUNTERS} 
and the profile output can be changed between a human readable version and a 
tab separated format using the flag \texttt{BSSN\_PROFILE\_HUMAN\_READABLE}.
Note that in order to profile communication, internal profile flags need to 
be enabled. The following is an example of profiling output for the first 
10 time steps. 
\begin{lstlisting}[basicstyle=\tiny]
active npes : 16
global npes : 16
current step : 10
partition tol : 0.1
wavelet tol : 0.0001
maxdepth : 12
Elements : 4656
DOF(zip) : 279521
DOF(unzip) : 2078609
============ MESH =================
step                 min(#)    mean(#)    max(#)                        
ghost Elements       634       824.062    1065                          
local Elements       263       291        319                           
ghost Nodes          43781     55671.7    71693                         
local Nodes          14292     17470.1    20705                         
send Nodes           18760     24872.9    36861                         
recv Nodes           18113     24872.9    33777                         
========== RUNTIME =================
step                 min(s)    mean(s)    max(s)                        
++2:1 balance        0         0          0                             
++mesh               1.9753    1.98299    1.98946                       
++rkstep             20.159    20.1856    20.1996                       
++ghostExchge.       1.81442   3.15703    4.49568                       
++unzip_sync         8.27839   9.67293    11.0991                       
++unzip_async        0         0          0                             
++isReMesh           0.04642   0.117357   0.207305                      
++gridTransfer       1.53709   1.54899    1.56531                       
++deriv              1.98942   2.34851    2.76695                       
++compute_rhs        4.00119   4.61547    5.11566                       
--compute_rhs_a      0.0137962 0.0245449  0.0351532                     
--compute_rhs_b      0.0296426 0.0503471  0.069537                      
--compute_rhs_gt     0.111898  0.12846    0.15463                       
--compute_rhs_chi    0.0170642 0.0315392  0.044856                      
--compute_rhs_At     2.40738   2.72922    3.05622                       
--compute_rhs_K      0.358215  0.39879    0.457139                      
--compute_rhs_Gt     0.774211  0.933581   1.05702                       
--compute_rhs_B      0.0575426 0.071209   0.0855094                     
++boundary con       0         0.0421986  0.134712                      
++zip                0.23529   0.260862   0.291513                      
++vtu                0.0872362 0.101362   0.128246                      
++checkpoint         3.27e-06  3.85e-06   5.7469e-06     
\end{lstlisting}

\subsection{Visualizing the data}
\dendro\  can be configured to output parallel unstructured grid files in 
binary file format (\texttt{.pvtu}).   
These files can be visualized using any visualization tool which supports 
VTK file formats. All the images presented in this paper used Paraview 
due to its robustness and scalability. Paraview allows python based 
scripting to perform \texttt{pvbatch} visualization, an example pvpython 
script can be found in \texttt{scripts/bssnVis.py}


%
%\newpage
%\pagebreak
%\setcounter{page}{1}
%\section{Artifact Evaluation}
%\label{sec:AE}
%\def\TT{{\rm time}}
\def\SS{{\rm step}}

In this section, we present experimental evaluations that we performed to ensure the accuracy of the simulation code. 

\subsection{Accuracy of stencil operators}

In order to test the accuracy and convergence of WAMR and the derivative stencils, we used a known function to generate adaptive octree grid based on the wavelet expansion. 
Then we compute numerical derivatives using finite difference stencils which are compared against the analytical derivatives of $f(x,y,z)$. $l_2$ and $l_{\infty}$ norms of the comparison is given in the Table \ref{tb:fd}. 

\begin{table}
	\centering
	\scalebox{0.9}{
		\begin{tabular}{||c | c |c | c||} 
			\hline
			derivative & grid points & $\norm{.}_2$ & $\norm{.}_\infty$ \\
			\hline\hline
			$\partial_x$ & 4913 & 0.0201773 & 0.00144632 \\
			$\partial_x$ & 99221 & 0.000849063 & 2.74672e-05 \\
			\hline
		\end{tabular}
	}
	\caption{Normed difference in numerical derivative and analytical derivative evaluated at grid points for the function $f(x,y,z)=sin(2\pi x)sin(2\pi y)sin(2\pi z))$% \  (x,y,z)\in [-0.5,0.5]^3$ 
	where in both cases wavelet tolerance 
	of $10^{-8}$ but increasing maximum level of refinement (i.e. \texttt{maxDepth}) from $4$ to $6$. Note that when \texttt{maxDepth} increases number of grid points increase hence normed difference between numerical and analytical derivatives goes down significantly. }
	\label{tb:fd}
\end{table}





\subsection{Accuracy of symbolic interface and code generation}

Given the complexity of the \BSSN\ equations, writing code to
evaluate these equations can be an error-prone and tedious task.
For example, to evaluate the equations we need to calculate more than 300
finite derivatives. Hence \dendro\ provides a symbolic framework
written in \texttt{SymPy} for automatically generating C++ code for
the equations. The user writes equations in a high-level representation 
that more closely resembles their symbolic form. 
We can then use the computational graph
of the equations to generate optimized C++ code. The accuracy
of the symbolic framework and code generation is certified by comparing 
results from the generated C++ code to those from 
the \HAD\ code, an established and tested code for numerical relativity.

This table shows a comparison of \BSSN\ equations (i.e. all 24 equations),
evaluated over a grid of $128^3$ points by arbitrary, non-zero functions
by both the \dendro\ and \HAD\ codes. All spatial derivatives in the equations
are evaluated using finite differences, for both the
\dendro\ and \HAD\ codes.
The table reports the $L_2$ norm of the difference in the equations
as evaluated in both codes, 
as well as the $L_2$ norms of the functions used to evaluate the equations.
Equations with residual norms of order $10^{-15}$ are clearly at machine zero, 
but this low level is reached for only the simplest equations. Residuals with
norms of order $10^{-12}$ arise in complicated equations, where finite precision  
errors can accumulate in hundreds of floating point operations. The optimized
equations require about 4500 floating point operations to evaluate at a single point.

\begin{lstlisting}[basicstyle=\tiny]
L2 Norms differences in the HAD and DENDRO equations on 128^3 points.
------------------------------------------------------------------------
|| diff 0  || = 1.18413e-15, || rhs HAD || = 1.28114, || rhs DENDRO || = 1.28114
|| diff 1  || = 7.53412e-16, || rhs HAD || = 2.18877, || rhs DENDRO || = 2.18877
|| diff 2  || = 4.98271e-16, || rhs HAD || = 1.66315, || rhs DENDRO || = 1.66315
|| diff 3  || = 1.03346e-15, || rhs HAD || = 0.720477,|| rhs DENDRO || = 0.720477
|| diff 4  || = 5.82489e-16, || rhs HAD || = 1.40142, || rhs DENDRO || = 1.40142
|| diff 5  || = 4.93128e-16, || rhs HAD || = 0.797567,|| rhs DENDRO || = 0.797567
|| diff 6  || = 1.94194e-11, || rhs HAD || = 19.0107, || rhs DENDRO || = 19.0107
|| diff 7  || = 2.14958e-11, || rhs HAD || = 19.5221, || rhs DENDRO || = 19.5221
|| diff 8  || = 4.0673e-12,  || rhs HAD || = 8.96364, || rhs DENDRO || = 8.96364
|| diff 9  || = 1.58532e-11, || rhs HAD || = 10.1459, || rhs DENDRO || = 10.1459
|| diff 10 || = 4.31184e-12, || rhs HAD || = 8.70053, || rhs DENDRO || = 8.70053
|| diff 11 || = 4.95696e-12, || rhs HAD || = 10.8644, || rhs DENDRO || = 10.8644
|| diff 12 || = 4.27211e-15, || rhs HAD || = 19.3546, || rhs DENDRO || = 19.3546
|| diff 13 || = 2.29542e-10, || rhs HAD || = 93.4829, || rhs DENDRO || = 93.4829
|| diff 14 || = 1.7484e-11,  || rhs HAD || = 40.0804, || rhs DENDRO || = 40.0804
|| diff 15 || = 4.79216e-11, || rhs HAD || = 30.9279, || rhs DENDRO || = 30.9279
|| diff 16 || = 2.03434e-11, || rhs HAD || = 26.0603, || rhs DENDRO || = 26.0603
|| diff 17 || = 1.07479e-15, || rhs HAD || = 15.5765, || rhs DENDRO || = 15.5765
|| diff 18 || = 3.00151e-15, || rhs HAD || = 32.9891, || rhs DENDRO || = 32.9891
|| diff 19 || = 7.79103e-16, || rhs HAD || = 8.10107, || rhs DENDRO || = 8.10107
|| diff 20 || = 7.77113e-16, || rhs HAD || = 9.01369, || rhs DENDRO || = 9.01369
|| diff 21 || = 1.74839e-11, || rhs HAD || = 46.3141, || rhs DENDRO || = 46.3141
|| diff 22 || = 4.79217e-11, || rhs HAD || = 32.7981, || rhs DENDRO || = 32.7981
|| diff 23 || = 2.03434e-11, || rhs HAD || = 32.7256, || rhs DENDRO || = 32.7256
\end{lstlisting}

\subsection{Single black hole}
\label{sec:AE_sbh}

Prior to simulating binary inspirals, we perform simple experiments
with a single black hole to ensure the accuracy of the simulation code. 
While a single black hole is a stable, static solution of the Einstein 
equations, although there is some transient time dependence with:w
 our particular 
coordinate conditions.
%The key idea of the
%single black hole experiment is we make the mass of the one black
%hole comparably smaller and make it's initial placement far away
%from the second black hole, which makes the binary black hole system
%reduce to the single black hole, and since there is no any interaction
%solution of the single black hole should remain as same as the
%initial solution as we perform time stepping.  
The black hole
parameters (parameter file can be found in
\texttt{SC18\_AE/par/single\_bh1.par} in the repository) for this
test is given below. Note that to generate initial data for a single black
hole, we place one of the black holes in the binary far from the computational
domain and set its mass to zero.

\begin{lstlisting}[basicstyle=\small]
"BSSN_BH1": { "MASS":1.0, "X":0.0, "Y":0.0, 
"Z": 0.00123e-6, "V_X": 0.0, "V_Y": 0.0, 
"V_Z": 0.0, "SPIN": 0, 
"SPIN_THETA":0, "SPIN_PHI": 0 },

"BSSN_BH2": { "MASS":1e-15,"X":1e15, "Y":0.0, 
"Z":0.00123e-6, "V_X": 0.0, "V_Y": 0.0, 
"V_Z": 0.0, "SPIN": 0, 
"SPIN_THETA":0, "SPIN_PHI": 0 }
\end{lstlisting}


\begin{figure*}
	\begin{minipage}{0.32\linewidth}
		\subfloat[{$\SS=0,\ \TT=0~M$}] {
			\includegraphics[width=0.8\textwidth]{figs/AE/sbh_boost/img_slice_000000.png}
		}
	\end{minipage}
	\begin{minipage}{0.32\linewidth}
		\subfloat[{$\SS=500,\ \TT=21.97~M$}] {
			\includegraphics[width=0.8\textwidth]{figs/AE/sbh_boost/img_slice_000050.png}
		}
	\end{minipage}
	\begin{minipage}{0.32\linewidth}
		\subfloat[{$\SS=1000,\ \TT=43.94~M$}] {
			\includegraphics[width=0.8\textwidth]{figs/AE/sbh_boost/img_slice_000100.png}
		}
	\end{minipage} \hfill

	\begin{minipage}{0.32\linewidth}
			\subfloat[{$\SS=1500,\ \TT=65.91~M$}] {
				\includegraphics[width=0.8\textwidth]{figs/AE/sbh_boost/img_slice_000150.png}
			}
		\end{minipage}
		\begin{minipage}{0.32\linewidth}
			\subfloat[{$\SS=2000,\ \TT=87.88~M$}] {
				\includegraphics[width=0.8\textwidth]{figs/AE/sbh_boost/img_slice_000200.png}
			}
		\end{minipage}
		\begin{minipage}{0.32\linewidth}
			\subfloat[{$\SS=2500,\ \TT=109.85~M$}] {
				\includegraphics[width=0.8\textwidth]{figs/AE/sbh_boost/img_slice_000250.png}
			}
		\end{minipage} \hfill
	
	\begin{minipage}{0.32\linewidth}
		\subfloat[{$\SS=3000,\ \TT=131.82~M$}] {
			\includegraphics[width=0.8\textwidth]{figs/AE/sbh_boost/img_slice_000300.png}
		}
	\end{minipage}
	\begin{minipage}{0.32\linewidth}
		\subfloat[{$\SS=3500,\ \TT=153.79~M$}] {
			\includegraphics[width=0.8\textwidth]{figs/AE/sbh_boost/img_slice_000350.png}
		}
	\end{minipage}
	\begin{minipage}{0.32\linewidth}
		\subfloat[{$\SS=4000,\ \TT=175.76~M$}] {
			\includegraphics[width=0.8\textwidth]{figs/AE/sbh_boost/img_slice_000400.png}
		}
	\end{minipage} \hfill
	
	
	\begin{minipage}{0.32\linewidth}
		\subfloat[{$\SS=4500,\ \TT=197.73~M$}] {
			\includegraphics[width=0.8\textwidth]{figs/AE/sbh_boost/img_slice_000450.png}
	}
	\end{minipage}
	\begin{minipage}{0.32\linewidth}
		\subfloat[{$\SS=5000,\ \TT=219.70~M$}] {
			\includegraphics[width=0.8\textwidth]{figs/AE/sbh_boost/img_slice_000500.png}
		}
	\end{minipage}
	\begin{minipage}{0.32\linewidth}
		\subfloat[{$\SS=5500,\ \TT=241.67~M$}] {
			\includegraphics[width=0.8\textwidth]{figs/AE/sbh_boost/img_slice_000550.png}
	}
	\end{minipage} \hfill
	
		\begin{minipage}{0.32\linewidth}
			\subfloat[{$\SS=6000,\ \TT=263.64~M$}] {
				\includegraphics[width=0.8\textwidth]{figs/AE/sbh_boost/img_slice_000600.png}
			}
		\end{minipage}
		\begin{minipage}{0.32\linewidth}
			\subfloat[{$\SS=6500,\ \TT=285.61~M$}] {
				\includegraphics[width=0.8\textwidth]{figs/AE/sbh_boost/img_slice_000649.png}
			}
		\end{minipage} \hfill
%		\begin{minipage}{0.32\linewidth}
%			\subfloat[{$\SS=550$}] {
%				\includegraphics[width=0.8\textwidth]{figs/AE/sbh_boost/img_slice_000550.png}
%			}
%		\end{minipage} \hfill
	
	\caption{A single black hole boosted in the $x$-direction, with \maxDepth=12 and wavelet tolerance of $10^{-3}$. Time is given in terms of the black hole
mass, $M$. \label{fig:sbh_boost}}
\end{figure*}





\begin{figure*}
	\begin{minipage}{0.32\linewidth}
		\subfloat[{$\SS=0,\ \TT=0~M$}] {
			\includegraphics[width=0.8\textwidth]{figs/AE/r1/img_slice_000000.png}
		}
	\end{minipage}
	\begin{minipage}{0.32\linewidth}
		\subfloat[{$\SS=500,\ \TT=21.97~M$}] {
			\includegraphics[width=0.8\textwidth]{figs/AE/r1/img_slice_000010.png}
		}
	\end{minipage}
	\begin{minipage}{0.32\linewidth}
		\subfloat[{$\SS=1000,\ \TT=43.94~M$}] {
			\includegraphics[width=0.8\textwidth]{figs/AE/r1/img_slice_000020.png}
		}
	\end{minipage} \hfill
	
	\begin{minipage}{0.32\linewidth}
		\subfloat[{$\SS=1500,\ \TT=65.91~M$}] {
			\includegraphics[width=0.8\textwidth]{figs/AE/r1/img_slice_000030.png}
		}
	\end{minipage}
	\begin{minipage}{0.32\linewidth}
		\subfloat[{$\SS=2000,\ \TT=87.88~M$}] {
			\includegraphics[width=0.8\textwidth]{figs/AE/r1/img_slice_000040.png}
		}
	\end{minipage}
	\begin{minipage}{0.32\linewidth}
		\subfloat[{$\SS=2500,\ \TT=109.85~M$}] {
			\includegraphics[width=0.8\textwidth]{figs/AE/r1/img_slice_000050.png}
		}
	\end{minipage} \hfill
	
	\begin{minipage}{0.32\linewidth}
		\subfloat[{$\SS=3000,\ \TT=131.82~M$}] {
			\includegraphics[width=0.8\textwidth]{figs/AE/r1/img_slice_000060.png}
		}
	\end{minipage}
	\begin{minipage}{0.32\linewidth}
		\subfloat[{$\SS=3500,\ \TT=153.79~M$}] {
			\includegraphics[width=0.8\textwidth]{figs/AE/r1/img_slice_000070.png}
		}
	\end{minipage}
	\begin{minipage}{0.32\linewidth}
		\subfloat[{$\SS=4000,\ \TT=175.76~M$}] {
			\includegraphics[width=0.8\textwidth]{figs/AE/r1/img_slice_000080.png}
		}
	\end{minipage} \hfill
	
	
	\begin{minipage}{0.32\linewidth}
		\subfloat[{$\SS=4500,\ \TT=197.73~M$}] {
			\includegraphics[width=0.8\textwidth]{figs/AE/r1/img_slice_000090.png}
		}
	\end{minipage}
	\begin{minipage}{0.32\linewidth}
		\subfloat[{$\SS=5000,\ \TT=219.70~M$}] {
			\includegraphics[width=0.8\textwidth]{figs/AE/r1/img_slice_000100.png}
		}
	\end{minipage}
	\begin{minipage}{0.32\linewidth}
		\subfloat[{$\SS=5500,\ \TT=241.67~M$}] {
			\includegraphics[width=0.8\textwidth]{figs/AE/r1/img_slice_000110.png}
		}
	\end{minipage} \hfill

	\caption{This figure shows frames from the evolution of a black hole
binary with an equal mass ratio, $q=1$. Time is measured in terms of the total
black hole mass $M$.\label{fig:r1}	}
\end{figure*}


\begin{figure*}
	\begin{minipage}{0.32\linewidth}
		\subfloat[{$\SS=0,\ \TT=0~M$}] {
			\includegraphics[width=0.8\textwidth]{figs/AE/r10/img_slice_000000.png}
		}
	\end{minipage}
	\begin{minipage}{0.32\linewidth}
		\subfloat[{$\SS=500,\ \TT=21.97~M$}] {
			\includegraphics[width=0.8\textwidth]{figs/AE/r10/img_slice_000010.png}
		}
	\end{minipage}
	\begin{minipage}{0.32\linewidth}
		\subfloat[{$\SS=1000,\ \TT=43.94~M$}] {
			\includegraphics[width=0.8\textwidth]{figs/AE/r10/img_slice_000020.png}
		}
	\end{minipage} \hfill
	
	\begin{minipage}{0.32\linewidth}
		\subfloat[{$\SS=1500,\ \TT=65.91~M$}] {
			\includegraphics[width=0.8\textwidth]{figs/AE/r10/img_slice_000030.png}
		}
	\end{minipage}
	\begin{minipage}{0.32\linewidth}
		\subfloat[{$\SS=2000,\ \TT=87.88~M$}] {
			\includegraphics[width=0.8\textwidth]{figs/AE/r10/img_slice_000040.png}
		}
	\end{minipage}
	\begin{minipage}{0.32\linewidth}
		\subfloat[{$\SS=2500,\ \TT=109.85~M$}] {
			\includegraphics[width=0.8\textwidth]{figs/AE/r10/img_slice_000050.png}
		}
	\end{minipage} \hfill
	
	\begin{minipage}{0.32\linewidth}
		\subfloat[{$\SS=3000,\ \TT=131.82~M$}] {
			\includegraphics[width=0.8\textwidth]{figs/AE/r10/img_slice_000060.png}
		}
	\end{minipage}
	\begin{minipage}{0.32\linewidth}
		\subfloat[{$\SS=3500,\ \TT=153.79~M$}] {
			\includegraphics[width=0.8\textwidth]{figs/AE/r10/img_slice_000070.png}
		}
	\end{minipage}
	\begin{minipage}{0.32\linewidth}
		\subfloat[{$\SS=4000,\ \TT=175.76~M$}] {
			\includegraphics[width=0.8\textwidth]{figs/AE/r10/img_slice_000080.png}
		}
	\end{minipage} \hfill
	
	
	\begin{minipage}{0.32\linewidth}
		\subfloat[{$\SS=4500,\ \TT=197.73~M$}] {
			\includegraphics[width=0.8\textwidth]{figs/AE/r10/img_slice_000090.png}
		}
	\end{minipage}
	\begin{minipage}{0.32\linewidth}
		\subfloat[{$\SS=5000,\ \TT=219.70~M$}] {
			\includegraphics[width=0.8\textwidth]{figs/AE/r10/img_slice_000100.png}
		}
	\end{minipage}
	\begin{minipage}{0.32\linewidth}
		\subfloat[{$\SS=5500,\ \TT=241.67~M$}] {
			\includegraphics[width=0.8\textwidth]{figs/AE/r10/img_slice_000100.png}
		}
	\end{minipage} \hfill
	

	\caption{This figure shows frames from the evolution of a black hole
binary with mass ratio $q=10$. Time is measured in terms of the total
black hole mass $M$.\label{fig:r10}}
\end{figure*}




\begin{figure*}
	\begin{minipage}{0.32\linewidth}
		\subfloat[{$\SS=0,\ \TT=0~M$}] {
			\includegraphics[width=0.8\textwidth]{figs/AE/r100/img_slice_000000.png}
		}
	\end{minipage}
	\begin{minipage}{0.32\linewidth}
		\subfloat[{$\SS=500,\ \TT=21.97~M$}] {
			\includegraphics[width=0.8\textwidth]{figs/AE/r100/img_slice_000020.png}
		}
	\end{minipage}
	\begin{minipage}{0.32\linewidth}
		\subfloat[{$\SS=1000,\ \TT=43.94~M$}] {
			\includegraphics[width=0.8\textwidth]{figs/AE/r100/img_slice_000040.png}
		}
	\end{minipage} \hfill

	\begin{minipage}{0.32\linewidth}
		\subfloat[{$\SS=1500,\ \TT=65.91~M$}] {
			\includegraphics[width=0.8\textwidth]{figs/AE/r100/img_slice_000060.png}
		}
	\end{minipage} 
	\begin{minipage}{0.32\linewidth}
		\subfloat[{$\SS=2000,\ \TT=87.88~M$}] {
			\includegraphics[width=0.8\textwidth]{figs/AE/r100/img_slice_000080.png}
		}
	\end{minipage}
	\begin{minipage}{0.32\linewidth}
		\subfloat[{$\SS=2500,\ \TT=109.85~M$}] {
			\includegraphics[width=0.8\textwidth]{figs/AE/r100/img_slice_000100.png}
		}
	\end{minipage} \hfill

	\begin{minipage}{0.32\linewidth}
		\subfloat[{$\SS=3000,\ \TT=131.82~M$}] {
			\includegraphics[width=0.8\textwidth]{figs/AE/r100/img_slice_000120.png}
		}
	\end{minipage}
	\begin{minipage}{0.32\linewidth}
		\subfloat[{$\SS=3500,\ \TT=153.79~M$}] {
			\includegraphics[width=0.8\textwidth]{figs/AE/r100/img_slice_000140.png}
		} 
	\end{minipage} 
	\begin{minipage}{0.32\linewidth}
		\subfloat[{$\SS=4000,\ \TT=175.76~M$}] {
			\includegraphics[width=0.8\textwidth]{figs/AE/r100/img_slice_000160.png}
		}
	\end{minipage} \hfill

	\begin{minipage}{0.32\linewidth}
		\subfloat[{$\SS=4500,\ \TT=197.73~M$}] {
			\includegraphics[width=0.8\textwidth]{figs/AE/r100/img_slice_000180.png}
		}
	\end{minipage}
	\begin{minipage}{0.32\linewidth}
		\subfloat[{$\SS=5000,\ \TT=219.70~M$}] {
			\includegraphics[width=0.8\textwidth]{figs/AE/r100/img_slice_000200.png}
		}
	\end{minipage}
	\begin{minipage}{0.32\linewidth}
		\subfloat[{$\SS=5500,\ \TT=241.67~M$}] {
			\includegraphics[width=0.8\textwidth]{figs/AE/r100/img_slice_000220.png}
		}
	\end{minipage}\hfill
 
	\caption{This figure shows frames from the evolution of a black hole binary 
with mass ratio $q=100$. Time is given
in terms of the total mass $M$.\label{fig:r100}}
\end{figure*}











\subsection{Boosted Single Black Hole}
\label{sec:AE_sbhboost}

The next experiment is an extension of the single BH test;
it ``boosts'' the BH with constant velocity in $x$-direction.
The constant velocity of the BH should be apparent in the evolution. 
The parameter file for this test can be found in the repository at 
\texttt{SC18\_AE/par/single\_bh1\_boost.par}.
The black hole parameters are given below (note
that $BSSN\_BH1$ has a momentum of 0.114 in $x$-direction).

\begin{lstlisting}[basicstyle=\small]
"BSSN_BH1": { "MASS":1.0, "X":0.0, "Y":0.0, 
"Z": 0.00123e-6, "V_X": 0.114, "V_Y": 0.0, 
"V_Z": 0.0, "SPIN": 0, 
"SPIN_THETA":0, "SPIN_PHI": 0 },

"BSSN_BH2": { "MASS":1e-15,"X":1e15, "Y":0.0, 
"Z":0.00123e-6, "V_X": 0.0, "V_Y": 0.0, 
"V_Z": 0.0, "SPIN": 0, 
"SPIN_THETA":0, "SPIN_PHI": 0 }
\end{lstlisting}



\subsection{Constraint equations}

Similar to the Maxwell equations of electrodynamics, the Einstein equations
contain both hyperbolic evolution equations and elliptic constraint equations,
which must be satisfied at all times. Following the common practice in
numerical relativity, we evolve the hyperbolic equations and monitor the 
quality of the solution by checking that the constraint equations are 
satisfied. The choice of coordinates for the BBH evolution (the puncture gauge) 
does induce constraint violations in the vicinity of each black hole.
The violations of the constraint equations in our runs are consistent with 
the discretization error expected for the numerical derivatives in the
constraint equations and the constraint violations near the black holes 
(punctures). An example of monitored constraint violations are listed in the Table \ref{tb:constraints}.

\begin{table}
	\centering
\scalebox{0.9}{
	\begin{tabular}{||c | c | c | c| c||} 
		\hline
		Time (M) & $\norm{\mathcal{H}_{r>a}}_2$ & $\norm{M1_{r>a}}_2$ & $\norm{M2_{r>a}}_2$ & $\norm{M3_{r>a}}_2$ \\
		\hline\hline
		0 & 0.000777861 & 1.01855e-05 & 1.23443e-05 & 7.17572e-06 \\
		0.976562 & 0.000808294 & 3.05681e-05 & 2.91217e-05 & 2.79631e-05 \\
		1.95312 & 0.000793783 & 5.03912e-05 & 3.93872e-05 & 4.06693e-05 \\
		2.92969 & 0.00079551 & 7.54643e-05 & 5.068e-05 & 5.42466e-05 \\
		3.90625	& 0.000956987 & 0.000102901 & 7.38208e-05 & 7.47156e-05 \\
		4.88281 & 0.00247348 & 0.000200055	& 0.000140583 & 0.000139008 \\
		\hline
	\end{tabular}
}
\caption{Violation of constraint equations with time for an equal mass ratio binary merger simulation done using OT. Note that $\mathcal{H}$, $M1,M2,M3$ denotes the Hamiltonian \& 3 momentum component constraints that is being monitored through the evolution.}
\label{tb:constraints}
\end{table}





\subsection{Binary black holes with mass ratio $q=1$}

We performed a series of short-term binary BH evolutions with different
mass ratios. This run was used to validate the code by comparing the
trajectories of the BHs calculated using \dendro\ to the trajectories calculated
by \HAD. Frames from the evolution are shown in the Figure~\ref{fig:r1},
%the figure,
and the BH parameters used for this run are listed below.

\begin{lstlisting}[basicstyle=\small]
  "BSSN_BH1": {
    "MASS":0.485,
    "X": 4.00e+00, "Y":0.0, "Z": 1.41-05,
    "V_X": -0.00133, "V_Y": 0.112, "V_Z": 0,
    "SPIN": 0, "SPIN_THETA":0, "SPIN_PHI": 0 },
  "BSSN_BH2": {
      "MASS":0.485,
      "X":-4.00+00, "Y":0.0, "Z":1.41-05,
      "V_X": 0.00132, "V_Y": -0.112, "V_Z": 0,
      "SPIN": 0, "SPIN_THETA":0, "SPIN_PHI": 0 }
\end{lstlisting}


\subsection{Binary black holes with mass ratio $q=10$}

We performed a short simulation with a mass ratio $q=10$
This is a short demonstration run to show that \dendro\
easily handles large mass ratios and gives consistent
results for the binary evolution.
Frames from the evolution are shown in figure,
and the BH parameters used for this run are listed below.

\begin{lstlisting}[basicstyle=\small]
  "BSSN_BH1": {
    "MASS":0.903,
    "X":5.45-01, "Y":0.0, "Z": 1.41-05,
    "V_X": -3.90e-04, "V_Y": 0.0470, "V_Z": 0,
    "SPIN": 0, "SPIN_THETA":0, "SPIN_PHI": 0 },
  "BSSN_BH2": {
      "MASS":0.0845,
      "X":-5.45+00, "Y":0.0, "Z":1.41-05,
      "V_X": 3.90e-04, "V_Y": -0.0470, "V_Z": 0,
      "SPIN": 0, "SPIN_THETA":0, "SPIN_PHI": 0 }
\end{lstlisting}

\subsection{Binary black holes with mass ratio $q=100$}

We performed a short simulation with a mass ratio $q=100$
This is a short demonstration run to show that \dendro\ 
produces the proper grid structure for this system and
reasonable results for a very challenging binary configuration.
Frames from the evolution are shown in the figure
and the BH parameters used for this run are listed below.


\begin{lstlisting}[basicstyle=\small]
  "BSSN_BH1": {
    "MASS":0.989,
    "X":5.94-02, "Y":0.0, "Z": 1.41-05,
    "V_X": -5.60-06, "V_Y": 5.61-03, "V_Z": 0,
    "SPIN": 0, "SPIN_THETA":0, "SPIN_PHI": 0 },
  "BSSN_BH2": {
      "MASS":0.00914,
      "X":-5.94+00, "Y":0.0, "Z":1.41421356e-05,
      "V_X": 5.60-06, "V_Y": -5.61-03, "V_Z": 0,
      "SPIN": 0, "SPIN_THETA":0, "SPIN_PHI": 0 }
\end{lstlisting}




% that's all folks
\end{document}


